```latex
\documentclass[a4paper,12pt]{article}
\usepackage{amsmath}
\usepackage{graphicx}
\usepackage{enumitem}

\title{Answer Sheet Matching}
\author{}
\date{}

\begin{document}

\maketitle

\section*{Matched Answers}

\begin{enumerate}[label=\textbf{Q\arabic*.}]
    \item \textbf{Short answer/objective question.}
    \begin{enumerate}[label=(\alph*)]
        \item \textbf{Answer:} \\
        \textit{Incidence Matrix to Adjacency Matrix}

        The incidence matrix provided is:
        \[
        \begin{bmatrix}
        1 & 0 & 0 \\
        1 & 1 & 1 \\
        0 & 1 & 0 \\
        0 & 0 & 1 \\
        \end{bmatrix}
        \]

        The matrix structure has been given the name. Each edge in an incidence matrix corresponds to a row and every node has its designated column. Each entry of value '1' shows that the node has an incidence with the edge. A simple undirected graph contains exactly two adjacent nodes for every single edge within it.

        From the incidence matrix:
        \begin{itemize}
            \item Edge 1 connects Node 1 and Node 2
            \item Edge 2 connects Node 2 and Node 3
            \item Edge 3 connects Node 2 and Node 4
        \end{itemize}

        The connection relationship between nodes in a graph structure appears in adjacency matrix format. The adjacency matrix contains value 1 in position (i,j) and (j,i) when node i and node j having an edge, otherwise both positions contain 0.

        The adjacency matrix will represent node to node connection.
        \[
        \begin{bmatrix}
        0 & 1 & 0 & 0 \\
        1 & 0 & 1 & 1 \\
        0 & 1 & 0 & 0 \\
        0 & 1 & 0 & 0 \\
        \end{bmatrix}
        \]
        
        \item \textbf{Answer:} \\
        \textit{Option-(B) Erdős–Rényi (Random Network) Model.}

        In the Erdős–Rényi model, the edges connect randomly because every edge formation remains isolated from other edges.
        
        \item \textbf{Answer:} \\
        \textit{Option-(C) Nash Equilibrium}

        Players reach a stable position known as Nash Equilibrium when none of them achieve higher rewards through independent strategy modifications.
        
        \item \textbf{Answer:} \\
        \textit{Option-(B) Assortative Mixing}

        Under the concept of assortative mixing, similar nodes establish connections with each other.
        
        \item \textbf{Answer:} \\
        \textit{Option-(D) Because it quantifies how often a node lies on the shortest paths between other nodes.}

        The measurement technique called Betweenness Centrality evaluates nodes based on their capacity to link other nodes because it determines information flow efficiencies.
        
        \item \textbf{Answer:} \\
        \textit{Option-(C) The presence of many nodes with very high degrees (hubs) that maintain connectivity.}

        Scale-free networks (hubs) operate through central nodes which serve as crucial connections, so their attack vulnerability becomes particularly pronounced.
        
        \item \textbf{Answer:} \\
        \textit{Option-(A) The number of intra-community edges is significantly higher than expected in a random network with the same degree sequence.}
        
        \item \textbf{Answer:} \\
        \textit{Option-(B) \(\frac{2}{5}\)}

        Jaccard Coefficient,

        Neighbors of X: A, B, C, D

        Neighbors of Y: C, D, B

        Intersection: C, D (2 nodes)

        Union: A, B, C, D, E (5 nodes)

        Jaccard Coefficient = \(\frac{\text{Intersection}}{\text{Union}} = \frac{2}{5} = 0.4\)
        
        \item \textbf{Answer:} \\
        \textit{Option-(B) LTM uses edge probabilities independently.}

        LTM uses a weighted sum of active neighbors compared to a node threshold.

        CM operates through independent edge probability, but LTM adopts a threshold-based method which takes the weighted sum of neighbors' impact.
        
        \item \textbf{Answer:} \\
        \textit{Option-(B) Because aggregating features from dissimilar neighbors can blur the node's own representative features, making classification harder.}

        When heterophily occurs during GCN and heterophily operations, it becomes difficult to classify nodes because they lose their individual characteristics through the features aggregation process from dissimilar neighbors.
    \end{enumerate}
    
    \item \textbf{Network analysis concepts for vaccination strategy.} \\
    \textit{Influenza Spread and Vaccination}

    For reducing influenza transmission to 5\% level, a combined approach of centrality measurement should be used to determine who gets vaccinated first.

    A high value of betweenness centrality reveals that the node functions as a "bridge" through which the disease transmits to numerous other connections. The immunization of these specific people breaks transmission pathways between different segments of the network.

    High degree centrality helps identify "hubs" which possess numerous connections since they have the potential to infect multiple other people.

    A computation between betweenness and degree centrality should be performed for every individual:
    \begin{itemize}
        \item A list of ranked people forms based on the centrality measurement result.
        \item The group of people selected for vaccination needs a strong ranking in both centrality measures.
        \item Additional individuals should be included for vaccination until 5\% coverage is reached if the top selected candidates do not suffice.
    \end{itemize}

    Justification:

    A combined approach delivers better results since betweenness centrality identifies network bridges which link different network sections while degree centrality distinguishes hubs that spread widely.

    Vaccinating key hubs and bridges provides maximum efficiency by controlling both small outbreaks and widespread distribution of the virus.
    
    \item \textbf{Improving 'suggested collaborators' feature.} \\
    \textit{Suggested Collaborators Feature}

    Link Prediction: System utilizes linking prediction algorithms that generate potential collaboration prediction through the analysis of current co-authorship and citation data.

    When researchers A and B demonstrate numerous joint authorship links with each other, a link prediction method could indicate potential collaboration between them.

    Through Node2Vec, the system creates vector representations of researchers based on network structural information, embedding that aligns with each other between researchers demonstrates similar research interest and collaboration potential.

    Homophily:

    Due to the natural human tendency of homophily, the system will provide recommendations related to researchers within similar academic fields.

    Due to encouraging diverse collaboration, the system requires a diversity metric within its recommendation functionality. Although the link prediction score may be slightly lower, the system will give priority to research links that combine fields from distinct areas.
    
    \item \textbf{Girvan-Newman algorithm.}
    \begin{enumerate}[label=(\alph*)]
        \item \textbf{Answer:} \\
        \textit{Girvan-Newman Core Idea}

        The Girvan-Newman algorithm detects communities through a recursive method which successively eliminates edges connecting different communities.
        
        \item \textbf{Answer:} \\
        \textit{Edge Betweenness Centrality}

        Edge Betweenness Centrality analyzes the number of times an edge lies between all pairs of nodes to discover these edges.

        Edges with high betweenness values link separate communities according to this method.
        
        \item \textbf{Answer:} \\
        \textit{Computational Limitation}

        The cost of computation increases substantially when calculating edge betweenness centrality on extensive networks, especially when they involve frequent edge removal operations.
        
        \item \textbf{Answer:} \\
        \textit{Louvain Method}

        The Louvain method functions as a scalable approach which uses greedy optimization to shift nodes between communities during its iterative process.
    \end{enumerate}
    
    \item \textbf{PageRank algorithm.}
    \begin{enumerate}[label=(\alph*)]
        \item \textbf{Answer:} \\
        \textit{PageRank Algorithm}

        According to PageRank, the importance of nodes depends on both the quantity and quality of incoming links passing through them.

        Page importance derives from other pages that link to it.
        
        \item \textbf{Answer:} \\
        \textit{Damping Factor (d)}

        During random web navigation, a surfer has a probability expressed through the damping factor (d) either to pause their link clicks or to move to an arbitrary page.
        
        \item \textbf{Answer:} \\
        \textit{Dangling Nodes}

        The lack of outgoing links from dangling nodes makes PageRank flow through the network. We handle this condition using an equal distribution among all network nodes.

        Each iteration distributes the rank of dangling nodes across all network nodes in an equal way.
    \end{enumerate}
    
    \item \textbf{Game theory on network edge.}
    \begin{enumerate}[label=(\alph*)]
        \item \textbf{Answer:} \\
        \textit{A strategy pair where players can improve their payoff by unilaterally changing their strategy.}

        Check each pair:

        \((U, A)\): Payoff (3,2)
        \begin{itemize}
            \item Player 1: Switch to L \(\rightarrow\) 2 < 3, no improve
            \item Player 2: Switch to B \(\rightarrow\) 1 < 2, no improve
            \item Nash Equilibrium.
        \end{itemize}

        \((U, B)\): Payoff (0,1)
        \begin{itemize}
            \item Player 1: Switch L \(\rightarrow\) 2 > 0, improve
            \item Not Nash
        \end{itemize}

        \((L, B)\): Payoff (2,3)
        \begin{itemize}
            \item Player 1: Switch to U \(\rightarrow\) 0 < 2, No improve
            \item Player 2: Switch to A \(\rightarrow\) 0 < 3, no improvement
            \item Nash Equilibrium.
        \end{itemize}

        Therefore, there are two pure strategy Nash equilibriums: \((U, A)\) and \((L, B)\).
        
        \item \textbf{Answer:} \\
        \textit{Player 1 plays U with probability p, L with probability 1-p.}

        Player 2's strategy:

        Expected payoff for Player 2 if choosing Strategy A:
        \[
        \mathbb{E}[A] = p \times 2 + (1-p) \times 0 = 2p
        \]

        Expected payoff for Player 2 if choosing Strategy B:
        \[
        \mathbb{E}[B] = p \times 1 + (1-p) \times 3 = p + 3(1-p) = p + 3 - 3p = 3 - 2p
        \]
        
        \item \textbf{Answer:} \\
        \textit{For Player 2:}

        \[
        \mathbb{E}[A] = 2p = 2 \times 0.7 = 1.4
        \]
        \[
        \mathbb{E}[B] = 3 - 2p = 3 - 2 \times 0.7 = 1.6
        \]

        Since \(\mathbb{E}[B] > \mathbb{E}[A]\), Player 2 could choose Strategy B.

        For Player 1: \((0.7 U, 0.3 L)\)

        Expected payoff = \(0.7 \times 0 + 0.3 \times 2 = 0.6\) when Player 2 chooses B.

        The expected outcome for Player 1 receives a payoff of 0.6 if Player 2 receives a payoff of 1.6.
    \end{enumerate}
    
    \item \textbf{Graph Neural Network feature update.} \\
    \textit{Graph Neural Network Feature Update}

    \begin{center}
    \includegraphics[width=0.5\textwidth]{graph_diagram.png}
    \end{center}

    Given,

    Directed edge \(A \rightarrow B\), \(C \rightarrow B\), \(D \rightarrow B\)

    Weighted matrix \(W = \begin{bmatrix} 0.5 & 0 \\ 0.1 & 0.2 \end{bmatrix}\)

    Step 1: Aggregate neighbor features.
    \[
    h_A^{(0)} + h_C^{(0)} + h_D^{(0)} = \begin{bmatrix} 1 \\ \end{bmatrix} + \begin{bmatrix} 3 \\ \end{bmatrix} + \begin{bmatrix} 3 \\ \end{bmatrix}
    \]
    \[
    h_{N(B)}^{(0)} = \frac{1}{3} \begin{bmatrix} 7 \\ 6 \end{bmatrix} = \begin{bmatrix} \frac{7}{3} \\ 2 \end{bmatrix}
    \]

    Step 2: Transform
    \[
    W \begin{bmatrix} \frac{7}{3} \\ 2 \end{bmatrix} = \begin{bmatrix} 0.5 & 0 \\ 0.1 & 0.2 \end{bmatrix} \begin{bmatrix} \frac{7}{3} \\ 2 \end{bmatrix}
    \]
    \[
    = \begin{bmatrix} 0.5 \times \frac{7}{3} + 0 \times 2 \\ 0.1 \times \frac{7}{3} + 0.2 \times 2 \end{bmatrix}
    \]
    \[
    = \begin{bmatrix} \frac{7}{6} \\ 0.1 \times \frac{7}{3} + 0.4 \end{bmatrix}
    \]
    \[
    = \begin{bmatrix} \frac{7}{6} \\ \frac{7}{30} + 0.4 \end{bmatrix}
    \]

    Step 3: Activate
    \[
    \text{ReLU} \begin{bmatrix} \frac{7}{6}, \frac{7}{30} + 0.4 \end{bmatrix} = \begin{bmatrix} \frac{7}{6}, \frac{7}{30} + 0.4 \end{bmatrix}
    \]

    Final Answer \(h_B^{(1)} = \begin{bmatrix} \frac{7}{6}, \frac{7}{30} + 0.4 \end{bmatrix}\).
    
\end{enumerate}

\end{document}
```