\documentclass[a4paper,12pt]{article}
\usepackage{amsmath, amssymb, graphicx, geometry}
\geometry{margin=1in}

\title{Answer Sheet Matching}
\author{}
\date{}

\begin{document}

\maketitle

\section*{Matched Answers}

\begin{enumerate}
    \item \textbf{Question 1}
    \begin{enumerate}
        \item \textbf{(a)} \\
        \textit{Student's Answer:} \\
        Convert the incidence Matrix to adjacency Matrix: \\
        Given Matrix: \\
        \[
        \begin{bmatrix}
        1 & 0 & 0 \\
        1 & 1 & 1 \\
        0 & 1 & 0 \\
        0 & 0 & 1 \\
        \end{bmatrix}
        \]
        Column 1: Connects Node 1 and Node 2 $\rightarrow$ edge between 1 and 2 \\
        Column 2: Connects Node 2 and Node 3, Edge between 2 \& 3. \\
        Column 3: Connects Node 2 \& Node 4 $\rightarrow$ Edge between 2 \& 4. \\
        So Adjacency Matrix: \\
        \[
        \begin{bmatrix}
        0 & 1 & 0 & 0 \\
        1 & 0 & 1 & 1 \\
        0 & 1 & 0 & 0 \\
        0 & 1 & 0 & 0 \\
        \end{bmatrix}
        \]

        \item \textbf{(b)} \\
        \textit{Student's Answer:} \\
        Which Network Model Assumes that edges are formed between pairs of Nodes with a uniform probability, independent of each other. \\
        Ans: (B) Erdős–Rényi (Random Network) \\
        Reason: This Model Connects Node pairs with uniform and independent probability.

        \item \textbf{(c)} \\
        \textit{Student's Answer:} \\
        In game theory, a situation where No player can improve Their Outcome By Unilaterally changing Their Strategy, given the strategies of other players is known as: \\
        Ans: C Nash Equilibrium \\
        Reason: This is when No player benefits By changing strategy unilaterally.
    \end{enumerate}

    \item \textbf{Question 2}
    \begin{enumerate}
        \item \textbf{(a)} \\
        \textit{Student's Answer:} \\
        The Tendency for Individuals in Social Network to associate and bond with similar others is defined as: \\
        Ans: Associative Mixing \\
        Reason: This describes preferences for connectivity with similar others.

        \item \textbf{(b)} \\
        \textit{Student's Answer:} \\
        Why Might betweenness centrality be a More relevant Measure than degree centrality for Identifying critical nodes in a Network transmitting information that Must follow specific paths? \\
        Ans: Because it Quantifies how often a Node lies on the shortest paths between other Nodes. \\
        Reason: Betweenness centrality highlights Nodes critical to information flows.
    \end{enumerate}

    \item \textbf{Question 3} \\
    \textit{Student's Answer:} \\
    A key finding about Scale-free Networks (like those generated by the Barabási-Albert Model) is their robustness to random Node failure but vulnerabilities to targeted attacks on hubs, What underlying property Best explains this? \\
    Ans: The presence of Many Nodes with Very high degrees (hubs) \\
    Reason: Hubs keeps the Network connected, their loss disrupts connectivity.

    \item \textbf{Question 4}
    \begin{enumerate}
        \item \textbf{(a)} \\
        \textit{Student's Answer:} \\
        In Community detection, Optimizing for high Modularity aims to find partitions where: \\
        Ans: The Number of Intra-Community edges is significantly higher than expected in a random Network with the same degree sequences. \\
        Reason: High Modularity reflects strong Community Structure.

        \item \textbf{(b)} \\
        \textit{Student's Answer:} \\
        Consider the two Nodes X, \& Y. The Neighbors of X are A, B, C, D. \& The Neighbors of Y are C, D, E. What is the Jaccard Co-efficient for link prediction between X \& Y? \\
        Ans: 2/5. \\
        Reason: Jaccard Co-efficient = $|C \cap D| / |C \cup D| = 2 / (5) = 2/5$.

        \item \textbf{(c)} \\
        \textit{Student's Answer:} \\
        In the Context of Information Cascade Models, how does the Activation Mechanism differ fundamentally between the Independent Cascade Model (ICM) and the Linear Threshold Model (LTM)? \\
        Ans: ICM uses edge probabilities; LTM uses weighted sum (vs) Threshold. \\
        Reason: ICM is probabilistic per edge; LTM checks if influence exceeds Threshold.
    \end{enumerate}

    \item \textbf{Question 5}
    \begin{enumerate}
        \item \textbf{(a)} \\
        \textit{Student's Answer:} \\
        A Standard graph Convolutional Network (GCN) aggregates information from a Node's immediate Neighbors. Why Might this standard Message-passing approach be suboptimal for tasks like Node Classification in Networks with high heterophily (where connected Nodes tend to be dissimilar)? \\
        Answer: Because aggregating features from dissimilar Neighbors can blur the Node's own identity representative features, Making classification harder. \\
        Reason: Heterophilic graph lead to poor GCN Performance due to Noisy Aggregation.

        \item \textbf{(b)} \\
        \textit{Student's Answer:} \\
        Problem Statement: \\
        Vaccinate 5\% of the population to Minimize Infections in a social contact Network using Network Analysis concept. \\
        SIR Model: \\
        S $\rightarrow$ Susceptible, I $\rightarrow$ Infected \\
        R $\rightarrow$ Recovered (includes Vaccinated Individuals) \\
        Goal: Select Critical Nodes to vaccinate (Preemptively Move to R) \\
        Strategy: Use two Network Measures: \\
        1) Degree Centrality: Nodes with the highest Number of direct connections \\
        Why? They can infect Many Neighbors Quickly.

        \item \textbf{(c)} \\
        \textit{Student's Answer:} \\
        Betweenness Centrality \\
        Measures how often a Node lies on the shortest paths between other Nodes \\
        Why? Nodes acting as bridges between Clusters can spread infections across the Network. \\
        Combined Approach: \\
        Rank all Nodes by a Combined Score (e.g., weighted sum of degree and betweenness) \& then select the top 5\% \\
        Diagram: \\
        \[
        \begin{array}{c}
        \text{Cluster A} \quad \text{Cluster B} \\
        \circ \quad \quad \quad \quad \circ \\
        \mid \quad \quad \quad \quad \mid \\
        \circ \quad \quad \quad \quad \circ \\
        \mid \quad \quad \quad \quad \mid \\
        \circ \quad \quad \quad \quad \circ \\
        \end{array}
        \]
        Vaccinating (X) cuts off inter-cluster transmissions.
    \end{enumerate}

    \item \textbf{Question 6}
    \begin{enumerate}
        \item \textbf{(a)} \\
        \textit{Student's Answer:} \\
        Game between two players are here shown in table \\
        Payoff Matrix: \\
        \[
        \begin{array}{c|c|c}
        & \text{Strategy A} & \text{Strategy B} \\
        \hline
        U (\text{Player 1}) & (3,2) & (0,1) \\
        L (\text{Player 2}) & (2,0) & (2,3) \\
        \end{array}
        \]
        here first value = player's 1 payoff \\
        Second value = player's 2 payoff.

        \item \textbf{(b)} \\
        \textit{Student's Answer:} \\
        Pure Strategy Nash Equilibrium \\
        A Nash equilibrium is where No player wants to change strategy unilaterally. \\
        Check each cell: \\
        $\rightarrow$ (U,A) : player 2 prefers B since 3 $>$ 2. \\
        (U,B) player 1 prefers L since 2 $>$ 0 \\
        (L,A) player 2 prefers B (since 3 $>$ 0) \\
        (L,B) Both players are Best-responding to each other $\Rightarrow$ Nash Equilibrium. \\
        Answer: only (L,B) is pure strategy Nash equilibrium.

        \item \textbf{(c)} \\
        \textit{Student's Answer:} \\
        Expected payoffs for player 2 are: \\
        Let player 1 play: \\
        U with probability P. \\
        L with probability (1-P). \\
        Payoff of player 2: \\
        If they Choose A: \\
        = 2P + 0 (1-P) = 2P. \\
        If they Choose B \\
        1P + 3(1-P) = P + 3 - 3P = 3 - 2P. \\
        For (P=0.7) \\
        If P=0.7 then. A : 2*0.7 = 1.4 \& \\
        B : 3-2*0.7 = 1.6 \\
        Here player 2 should be choose strategy B, as it gives higher expected pay off. \\
        Here the Table.
    \end{enumerate}

    \item \textbf{Question 7} \\
    \textit{Student's Answer:} \\
    GNN Layer update for Node B. \\
    Problem Statement: The gives directed graph where Nodes A, C and D point to Node B, Here we need to calculate the updated feature Vectors \( h^{(1)}_B \) using simple GNN Layer \\
    Soln: \\
    GNN Layer operations \\
    $\Rightarrow$ Aggregate Neighbors features (Average) \\
    $\Rightarrow$ Transform using a weight Matrix \\
    Activate using ReLU \\
    Given Data: \\
    Initial feature Vectors: \\
    \[
    \begin{align*}
    h^{(0)}_A &= \begin{bmatrix} 1 \end{bmatrix} \\
    h^{(0)}_C &= \begin{bmatrix} 0 \\ 3 \end{bmatrix} \\
    h^{(0)}_D &= \begin{bmatrix} 2 \\ 2 \end{bmatrix} \\
    \end{align*}
    \]
    Neighbors of A, C, D. \\
    Weight Matrix W: \\
    \[
    W = \begin{bmatrix} 0.5 & 0 \\ 0.1 & 0.2 \end{bmatrix}
    \]
    Aggregate Neighbors feature (Average): \\
    \[
    h^{(0)}_N (B) = \frac{1}{3} \left( h^{(0)}_A + h^{(0)}_C + h^{(0)}_D \right) = \frac{1}{3} \begin{bmatrix} 1 + 0 + 2 \\ 1 + 3 + 2 \end{bmatrix} = \frac{1}{3} \begin{bmatrix} 3 \\ 6 \end{bmatrix}
    \]
    Linear Transformation: \\
    \[
    W \cdot \begin{bmatrix} 3 \\ 6 \end{bmatrix} = \begin{bmatrix} 0.5 \times 1 + 0.2 \times 2 \\ 0.1 \times 1 + 0.2 \times 2 \end{bmatrix} = \begin{bmatrix} 0.5 \\ 0.5 \end{bmatrix}
    \]
    Apply ReLU Activation: \\
    \[
    \text{ReLU}(x) = \max(0, x) \text{ applied element wise}
    \]
    \[
    h^{(1)}_B = \text{ReLU} \left( \begin{bmatrix} 0.5 \\ 0.5 \end{bmatrix} \right) = \begin{bmatrix} 0.5 \\ 0.5 \end{bmatrix}
    \]
    So the final Answer: \\
    \[
    h^{(1)}_B = \begin{bmatrix} 0.5 \\ 0.5 \end{bmatrix}
    \]
    This is the updated feature Vectors for Node B after one GNN Layer.
\end{enumerate}

\end{document}